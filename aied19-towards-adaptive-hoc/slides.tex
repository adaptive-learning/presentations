\documentclass[bigger]{beamer}

\usepackage{booktabs}

\usetheme{metropolis}
\metroset{block=fill}

\title{Towards Adaptive Hour of Code}

\author{Tom\'a\v{s} Effenberger\\[4mm]
%Masaryk University Brno\\
%Czech Republic
\includegraphics[width=.3\linewidth]{figures/al-logo}\\[6mm]
}

\newcommand{\img}[2]{
  \begin{center}
    \includegraphics[width=#1\linewidth]{figures/#2}
  \end{center}
}


\date{AIED 2019\\Doctoral Consortium}

\begin{document}

\frame{\titlepage}

\begin{frame}
  \frametitle{Introductory Programming}
  % High-level motivation + context:

% - HoC = popular way to introduce children into programming, used by millions
% - but: not personalized
% - aim: adaptive HoC -> more efficient and engaging learning

  \img{0.6}{robomission-on-yellow-to-left}

\end{frame}


\begin{frame}
  \frametitle{Research Questions}
  % "expected contribution to AIED" = answering these RQs

  \emph{(in the context of introductory programming)}
  % 4 high-level RQs, one concerning DM, PM, SM, and TM.
  \begin{enumerate}
  % [domain] How to organize tasks for a personalized Hour of Code  activity?
  \item How to organize tasks for a personalized Hour of Code?  % TODO: better RQ?
        % ~ what is a suitable domain model?
        % e.g., hierarchical linearly ordered PS (our hypothesis)

  % [performance] How to measure students' performance on programming tasks?
  \item How to measure performance on programming tasks?
    % Binary success not enough.
    % Other performance aspects: time, #edits, #executions, quality of program,
    % or event the full time series of created/executed programs.

  % [student] How to predict a future performance of a student on introductory programming tasks?
  \item How to predict future performance?
  % - Hypothesis: For use in outer loop (online), simple model enough (if good DM and PM).
  % - more complex model such as LogisticHMM useful for offline analysis (setting parameters).

  % [tutor] How to recommend the next task to practice in Hour of Code activities?
  \item How to recommend the next task to practice?
  % - mastery learning, adapted to non-binary performance and specific domain model
  % - random choice within a (sub)level to maximize (safe) exploration
  \end{enumerate}

  % Answering RQ1-3 are partially steps towards RQ4, but they are also important
  % on their own, e.g. for presentation of problem

  % These are sort of "standard questions", but they have not been answered
  % sufficiently well for the context of intro programming, e.g., most
  % research: binary performance, only 2 problems - we have several programming
  % exercises with tens of problems + possibility to apply interventions in our
  % systems.  My goal: answer them such that the results can be applied, e.g.,
  % in Hour of Code, KA.
\end{frame}


\begin{frame}
  \frametitle{Theoretical Framework}

  \begin{itemize}
  % Proxy goal for learning and engagement: optimal challenge:
  \item state of flow, zone of proximal development
  % + cognitive load theory, scaffolding

  % This is rather a "technical framework".
  % Decomposition of the problem/ALS:
  \item models: domain, performance, student, tutor

  % Adaptivity at different time scales:
  % - outer loop -- adapt to a particular student
  % - design loop -- improve the whole system (adapting to a population)
  \item adaptivity: outer loop (mastery learning), design loop % (human-in-the-loop)
  \end{itemize}

  % Just an illustration of an example of the models currently used in RoboMission.
  % TODO(maybe): Maybe remove if there is not enough time / simplify the diagram.
  \img{0.9}{robomission-tutor-model}

\end{frame}


\begin{frame}
  \frametitle{Methods}

  \begin{itemize}
  \item exploratory analysis, multiple exercises
    \begin{itemize}
    \item interface: Blockly x Python
    \item robot on grid, turtle graphics, numbers/text processing
    \end{itemize}
    % The exploratory analysis is not purely observational, because
    % we can also change the system (e.g. domain model) and evaluate
    % the intervention (design loop).
    % Multiple exercises: important for generalizability of results.

  \item online experiments to compare tutor models
    % "quasi-experiment?
    \begin{itemize}
    \item proxy for learning: performance on \emph{control tasks}
    \item chosen randomly after each problem set
      % Point for discussion: from which subset?
    \end{itemize}

  \item simulated experiments to explore methodological issues
    % Biases: (mastery) attrition, self-selection, learning, adaptive rcm, ...
    % How do biases influence the observed data and results?
  \end{itemize}

\end{frame}


%\begin{frame}
%  \frametitle{Summary}
%  \begin{itemize}
%  % Context, Motivation: efficient & engaging HoC
%  \item adaptive learning of introductory programming
%
%  % Framework
%  \item flow, mastery learning, design loop
%
%  % Contribution to AIED: answering the research questions concerning:
%  \item domain, performance, student, and tutor modeling
%  % - in the context of introductory programming
%  % - focus on evaluation and methodological issues
%  % - mention validation of replicability ("Expected contributions" box below)
%
%  % Methods:
%  \item exploratory analysis, online and simulated experiments
%  \end{itemize}
%
%  \begin{block}{Expected contributions}
%  \begin{itemize}
%  \item recommendations on modeling approaches and evaluation methods in the
%  context of introductory programming
%  \item replicability of previous results on a broader set of exercises and problems
%  \end{itemize}
%  \end{block}
%  % Stress the number of exercises and range of problem difficulties.
%  % Request: rcmd on previous studies (both their own and from others)
%  % that they maybe don't believe and would like to know if the results
%  % generalize.
%\end{frame}


\begin{frame}
  \frametitle{Expected contributions to AIED}
  % (also serves as a summary)
  \begin{itemize}
  % Context, Motivation: efficient & engaging HoC

  \item recommendations on modeling approaches\\
        in the context of introductory programming
    % domain, performance, student, and tutor modeling
    % = ~ answering the research questions concerning
  \item focus on evaluation and methodological issues
  \item replicability of previous results on a broader set of exercises and tasks
    % Stress the number of exercises and range of problem difficulties.
    % Request: rcmd on previous studies (both their own and from others)
    % that they maybe don't believe and would like to know if the results
    % generalize.
  \end{itemize}
\end{frame}

\end{document}
