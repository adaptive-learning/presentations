\documentclass[bigger]{beamer}

\usepackage{booktabs}

\usetheme{metropolis}
\metroset{block=fill}
\setbeamercolor{background canvas}{bg=white}

%\setbeamerfont{block title}{size={}}
%\beamertemplatenavigationsymbolsempty
%\setbeamertemplate{footline}{%
%   \raisebox{5pt}{\makebox[\paperwidth]{\hfill\makebox[10pt]{%
%   \color{gray}\scriptsize\insertframenumber}}}}

\title{Blockly Programming Dataset}

\author{Tom\'a\v{s} Effenberger\\[4mm]
%Masaryk University Brno\\
%Czech Republic
\includegraphics[width=.3\linewidth]{al-logo}\\[6mm]
}

\newcommand{\img}[2]{
  \begin{center}
    \includegraphics[width=#1\linewidth]{figures/#2}
  \end{center}
}


\date{CSEDM 2019}

\begin{document}

\frame{\titlepage}
% - My goal in the next 5 minutes is to present you a dataset
%   from a system for learning introductory programming
%   and to help you decide whether it could be useful for you
%  (... which is a euphemism for "persuade" ...)

\begin{frame}
  \frametitle{Block-Based Programming}

% - Block-based programming activities like HoC are a popular way to
%   introduce children into programming.
% - Currently, only few datasets available that could researchers use
%   to study modeling approaches in this context.
% - There is HoC dataset x only 2 tasks (still interesting research)
% - Single exercise limits generalizability
% - -> that's why we have decided to publish a dataset from data from
%   RoboMission, a block-based programming game developed in our lab
%   In RoboMission, students create programs for a spaceship to guide
%   it through the space, avoid obstacles and collect diamonds.
% - The game was designed so that it allows for a variety of difficulties
%   and for practice basic programming concepts such as sequences of commands,
%   loops and conditional commands.

  \img{1.0}{robomission-tasks}

\end{frame}


\begin{frame}
  \frametitle{RoboMission}

  % We tried to make the dataset useful not only for us, but also for
  % other researchers. So, to understand the data and how they were collected:
  \begin{itemize}
  \item freely available online game
    % ~ you can try the game
  \item description of system behavior and data collection
    % ~ you can read about the used models and algorithms in published papers
  \item open source code
    % ~ you can see implementation of the models
  \item broad range of difficulties  % from easy to complex
  \item 85 tasks  % divided into 9 levels
    % ~ you can explore learning dynamics across many tasks of diverse difficulty
  \end{itemize}
\end{frame}


\begin{frame}
  \frametitle{Data}

  % collected over 15 months

  \begin{itemize}
  \item 1.4 million program snapshots  (edit / execution) % "logged at each edit and execution"
    % OMIT: CSV file, format similar to ProgSnap2
    % OMIT: time, student, task, granularity (edit, execution), success
  % Omit all technical details, just mention that there are some aux files, e.g.
  % describing setting and a sample solution of tasks and that details can be
  % found in the documentation.
  \item 85\,000 task session (student-task pairs)  % OMIT
    % OMIT: aggregated
    % OMIT: student, task, start, end, success, time, number of edits/execs
  \item 85 tasks
    % allows to explore learning dynamics beyond a single task
    % setting and sample solution
    % OMIT: + problem sets csv: mapping between tasks and levels, common setting
  \item 5\,800 students  % OMIT
  \end{itemize}

  \bigskip
  \begin{block}{Dataset Documentation}
  \scriptsize{%
    \href{https://github.com/adaptive-learning/adaptive-learning-research/tree/master/data/robomission-2019-02-09}{\texttt{%
    https://github.com/adaptive-learning/adaptive-learning-research\\
    /tree/master/data/robomission-2019-02-09}}}
  \end{block}


\end{frame}

\begin{frame}
  \frametitle{Grid World and Code Representation}

  \img{1.}{task-setting-solution}

  % OMIT: - world: fields, unique letters for background colors (lower) and objects (upper)
  % - simple and compact text representation for both task sestting and submitted programs
  % - single letter for each keyword or action, with a context independent meaning
  % - [TRANSITION] can be expanded to Python, but these short mini-codes are useful
  %   e.g. because they fit as labels in a graph
\end{frame}


\begin{frame}
  \frametitle{Usage Example: Exploring State Space Graphs}

  \img{1.}{statespace-clean-path}

% - check for the code quality (e.g., detecting unexpected solutions)
% - prioritization, e.g. hint generation for the more problematic states
% - student performance measurement (did he do at least a few steps towards a solution?)
% - similarity of tasks, difficulty of tasks
% - EXP: illustration of logging granularity
% - EXP: illustration of variability of students' trajectories in a single "simple" task
\end{frame}

\begin{frame}
  \frametitle{Usage Examples}

  % We have already used it for:
  \begin{itemize}
  \item exploring state space graphs  % "transitions between code states in state space graphs"
  \item measuring similarity of programming tasks
  \item measuring performance on programming tasks
  \item measuring difficulty of programming tasks
  \end{itemize}
\end{frame}


\begin{frame}
  \frametitle{Summary}

  \begin{itemize}
  % \item block-based introductory programming
  \item 1.4 million program snapshots, 85 tasks  % fine-grained data % (each edit)
  % We strive to make the dataset useful:
  \item comprehensive data documentation
  \item freely available online game
  \item description of system behavior and data collection
  \item open source code
  \end{itemize}
  % You are welcome to download and use the dataset for your own research.


  \bigskip
  \begin{block}{Get the Data}
  \scriptsize{%
    \href{https://github.com/adaptive-learning/adaptive-learning-research/tree/master/data/robomission-2019-02-09}{\texttt{%
    https://github.com/adaptive-learning/adaptive-learning-research\\
    /tree/master/data/robomission-2019-02-09}}}
  \end{block}

\end{frame}

\end{document}
