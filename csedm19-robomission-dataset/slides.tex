\documentclass[bigger]{beamer}

\usepackage{booktabs}

\usetheme{metropolis}
\metroset{block=fill}
%\useinnertheme{circles}
%\usecolortheme{crane} % use default

%\setbeamerfont{block title}{size={}}
%\beamertemplatenavigationsymbolsempty
%\setbeamertemplate{footline}{%
%   \raisebox{5pt}{\makebox[\paperwidth]{\hfill\makebox[10pt]{%
%   \color{gray}\scriptsize\insertframenumber}}}}

\title{Blockly Programming Dataset}

\author{Tom\'a\v{s} Effenberger\\[4mm]
%Masaryk University Brno\\
%Czech Republic
\includegraphics[width=.3\linewidth]{al-logo}\\[6mm]
}

\newcommand{\img}[2]{
  \begin{center}
    \includegraphics[width=#1\linewidth]{figures/#2}
  \end{center}
}


\date{CSEDM 2019}

\begin{document}

\frame{\titlepage}
% - popular way to introduce children into programming
% - researchers need data
% - there is HoC dataset x only 2 tasks (still interesting research)
% - single exercise limits generalizability

\begin{frame}
  \frametitle{Block-Based Programming}

% - dataset from data from RoboMission, a block-based programming game
%   in which students write programs for a spaceship in space
% - describe the task in the screenshot

% Programming Game:
% - grid world
% - goal
% - game elements
% - block based programming
% - execution at any time

  \img{0.6}{robomission-on-yellow-to-left}

\end{frame}


\begin{frame}
  \frametitle{RoboMission}

  % Advantages of the dataset:
  \begin{itemize}
  \item online learning system  % you can try the game
  \item published models and algorithms  % you can read abou the used models and algorithms
  \item open source code % you can see implementation of the models
  \item 9 levels, 85 tasks  % divided into 9 levels
  % recommendation, mastery decision
  \item sequential execution, loops, conditions
  \item block-based programming
  \end{itemize}
\end{frame}


\begin{frame}
  \frametitle{Data}

  % collected over 15 months

  \begin{itemize}
  \item 1.4 million program snapshots  (events) % logged at each edit and execution
    % CSV file, format similar to ProgSnap2
    % time, student, task, granularity (edit, execution), success
  \item 85\,000 task session (student-task pairs)
    % aggregated
    % student, task, start, end, success, time, number of edits/execs
  \item 85 tasks
    % allows to explore learning dynamics beyond a single task
    % setting and sample solution
    % + problem sets csv: mapping between tasks and levels, common setting
  \item 5\,800 students
  \end{itemize}

\end{frame}

\begin{frame}
  \frametitle{Grid World and Code Representation}

  \img{1.}{task-setting-solution}

  % - world: fields, unique letters for background colors (lower) and objects (upper)
  % - solution: MiniCode single line, unique letters (capital for control structures)
\end{frame}


\begin{frame}
  \frametitle{Usage Example: Exploring State Space Graphs}

  \img{1.}{statespace-clean-path}

% - check for the code quality (e.g., detecting unexpected solutions)
% - prioritization, e.g. hint generation for the more problematic states
% - student performance measurement (did he do at least a few steps towards a solution?)
% - similarity of tasks, difficulty of tasks
\end{frame}

\begin{frame}
  \frametitle{Usage Examples}

  % We have already used it for:
  \begin{itemize}
  \item exploring state space graphs  % "transitions between code states in state space graphs"
  \item measuring similarity of programming tasks
  \item measuring performance on programming tasks
  \item measuring difficulty of programming tasks
  \end{itemize}
\end{frame}


\begin{frame}
  \frametitle{Summary}

  % Remind the advantages:
  \begin{itemize}
  \item block-based introductory programming
  \item 85 tasks
  \item 1.4 program snapshots  % high-granularity data % (each edit)
  \item comprehensive documentation  %
  \item free online game
  \item published models
  \item open source code
  \end{itemize}
\end{frame}

\end{document}
