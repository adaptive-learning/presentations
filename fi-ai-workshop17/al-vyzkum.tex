\documentclass[bigger]{beamer}

\usepackage{czech}
\usepackage[utf8]{inputenc}
\usepackage{booktabs}
\useinnertheme{rounded}
\usecolortheme{crane}
\setbeamerfont{block title}{size={}}

\newcommand{\img}[2]{\begin{center}\includegraphics[width=#1\linewidth]{#2}\end{center}}

\title{Adaptive Learning\\Přehled výzkumu a vývoje}

\author{Radek Pelánek\\[10mm]
  \includegraphics[width=.2\linewidth]{al-logo-researchgroup}}

\date{2017}

\begin{document}

\frame{\titlepage}

\begin{frame}
  \img{1}{slepemapy}
\end{frame}

\begin{frame}
  \img{.9}{anatom}
\end{frame}

\begin{frame}
  \img{.95}{poznavacka}
\end{frame}

\begin{frame}
  \img{.8}{umimecesky}
\end{frame}

\begin{frame}
  \img{.95}{matmat}
\end{frame}

\begin{frame}
  \img{.95}{robomise}
\end{frame}

\begin{frame}
  \frametitle{Předmět zájmu}

  obecný cíl: \emph{efektivnější výuka díky využití dat}

  \bigskip

  společné prvky činností:
  \begin{itemize}
  \item vývoj prakticky používaných výukových systémů
  \item systémy ukládají data
  \end{itemize}

  \bigskip

  využití dat:
  \begin{itemize}
  \item automatizovaná personalizace, adaptivita
  \item analýzy usměrňující vývoj
  \end{itemize}
\end{frame}

\begin{frame}
  \frametitle{Data}

  nejpoužívanější systémy (Slepé mapy, Umíme česky):
  \begin{itemize}
  \item uživatelé:
    \begin{itemize}
    \item denně: $\sim$ 1000 návštěvníků, 50000 odpovědí
    \item celkově: desítky milionů odpovědí
    \end{itemize}
  \item rozsah: $\sim$ 10\,000 položek
  \end{itemize}
\end{frame}

\begin{frame}
  \frametitle{Nejrelevantnější výzkumné komunity}

  \begin{itemize}
  \item Educational Data Mining
  \item Learning Analytics
  \item Artificial Intelligence in Education
  \end{itemize}
\end{frame}

\begin{frame}
  \frametitle{Výzkum}

  \begin{itemize}
  \item modelování znalostí studentů
  \item analýzy výukových otázek (struktura, obtížnost)
  \item algoritmy pro konstrukci otázek
  \item \textbf{evaluace} modelů, systémů
  \end{itemize}

  \bigskip

  strojové učení, statistika, zpracování dat, inspirace z kognitivních a
  pedagogických věd
\end{frame}

\begin{frame}
  \frametitle{Přesah i mimo výukové aplikace}

  \begin{itemize}
  \item evaluace -- obecné metodické problémy
  \item doporučující systémy
  \item podobnost položek
  \end{itemize}
\end{frame}

\begin{frame}
  \frametitle{Příklady výzkumu I}

  \emph{Elo-based Learner Modeling for the Adaptive Practice of Facts} (Journal
  of User Modeling and User-Adapted Interaction, 2017) 

  \bigskip

  \begin{itemize}
  \item Elo systém pro hodnocení šachistů
  \item modifikace systému pro využití ve výukových systémech
  \end{itemize}
\end{frame}


\begin{frame}
  \frametitle{Příklady výzkumu II}

  \emph{Metrics for Evaluation of Student Models} (Journal of Educational Data
  Mining, 2015)

  \bigskip

  \begin{itemize}
  \item modely predikují, zda student odpoví správně/špatně
  \item jak měřit kvalitu těchto modelů?
  \item metriky: RMSE, MAE, AUC, log-likelihood, accuracy\ldots
  \end{itemize}
\end{frame}

\begin{frame}
  \frametitle{Příklady výzkumu III}

  \emph{Impact of Data Collection on Interpretation and Evaluation of Student
    Models} (Learning Analytics \& Knowledge, 2016)

  \bigskip

  \begin{itemize}
  \item metodické aspekty sběru dat a evaluace
  \item zpětné vazby mezi modely a sběrem dat
  \end{itemize}
\end{frame}

\begin{frame}
  \frametitle{Příklady výzkumu IV}

  \emph{Evaluation of an Adaptive Practice System for Learning Geography Facts}
  (Learning Analytics \& Knowledge, 2016)

  \emph{Impact of Adaptive Educational System Behaviour on Student Motivation } (Artificial Intelligence in Education, 2015)

  \bigskip

  \begin{itemize}
  \item AB experimenty
  \item porovnání algoritmů pro výběr otázky
  \item vyhodnocení: dopad na motivaci a efektivitu učení
  \end{itemize}
\end{frame}

\begin{frame}
  \frametitle{Slepé mapy: křivky přežití}

  \img{.95}{survival_curve_by_ab}
\end{frame}

\begin{frame}
  \frametitle{Slepé mapy: učicí křivky}

  \img{.95}{target_difficulty_context_learning_slope}
\end{frame}

\begin{frame}
   \frametitle{Odpoledne: Ukázky systémů a analýz}

   \begin{itemize}
   \item \url{slepemapy.cz}
   \item \url{umimecesky.cz}
   \item \url{umimematiku.cz}
   \item \url{umimeanglicky.cz}
   \item \url{poznavackaprirody.cz}
   \item \url{anatom.cz}
   \item \url{robomise.cz}
   \end{itemize}
---
\medskip

\emph{Bezvadná věc. Škoda, že už nechodím do školy...}

\medskip
---
\medskip

\emph{...díky této hře jsem nejlepší ve třídě a jsem teď hodně chytrá}

\end{frame}

\end{document}