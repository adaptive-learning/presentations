\documentclass[bigger]{beamer}

\usepackage{czech}
\usepackage[utf8]{inputenc}
\usepackage{booktabs}
\useinnertheme{rounded}
\usecolortheme{crane}
\setbeamerfont{block title}{size={}}

\newcommand{\img}[2]{\begin{center}\includegraphics[width=#1\linewidth]{#2}\end{center}}

\title{Adaptive Learning\\Ukázky vyvinutých systémů a analýz}

\author{Radek Pelánek\\[10mm]
  \includegraphics[width=.2\linewidth]{al-logo-researchgroup}}

\date{2017}

\begin{document}

\frame{\titlepage}

\begin{frame}
  \frametitle{Ukázky systémů}

  \begin{itemize}
  \item \textbf{Slepé mapy} (další podobné: Anatom, Poznávačka přírody)
  \item \textbf{Umíme česky} (další podobné: Umíme anglicky, Umíme matiku)
  \item \textbf{Robomise}
  \end{itemize}
\end{frame}

\begin{frame}
  \frametitle{Slepé mapy}

  \texttt{slepemapy.cz}

  \bigskip

  \begin{itemize}
  \item dostupný obsah
  \item podoba otázek
  \end{itemize}
\end{frame}

\begin{frame}
  \frametitle{Slepé mapy: AB experiment}

  TODO
\end{frame}

\begin{frame}
  \frametitle{Umíme česky}

  \texttt{umimecesky.cz}

  \bigskip

  \begin{itemize}
  \item strom konceptů, mapa učiva
  \item míra znalostí, \uv{mastery learning}
  \item netradiční hry: Střílečka, Tetris, hry pro více hráčů
  \end{itemize}
\end{frame}

\begin{frame}
  \frametitle{Robomise}

  \texttt{robomise.cz}

  \bigskip

  \begin{itemize}
  \item kontext: Hour of code
  \item základní prostředí postavené na Blockly
  \item inovativní pojetí úlohy
  \item adaptivní chování
  \end{itemize}
\end{frame}

\begin{frame}
  \frametitle{Ukázky analýz}

  \begin{itemize}
  \item podobnost položek
  \item křivky přežití, křivky učení
  \item vyhodnocení kritérií zvládnutí
  \end{itemize}
\end{frame}

\begin{frame}
  \frametitle{Podobnost položek}

  TODO obrázky
\end{frame}

\begin{frame}
  \frametitle{Křivky přežití, křivky učení}

  TODO obrázky
\end{frame}

\begin{frame}
  \frametitle{Vyhodnocení kritérií zvládnutí}

  TODO obrázky
\end{frame}


\end{document}