\documentclass[bigger]{beamer}

\usepackage{booktabs}

\usetheme{metropolis}
\metroset{block=fill}
\setbeamercolor{background canvas}{bg=white}

%\setbeamerfont{block title}{size={}}
%\beamertemplatenavigationsymbolsempty
%\setbeamertemplate{footline}{%
%   \raisebox{5pt}{\makebox[\paperwidth]{\hfill\makebox[10pt]{%
%   \color{gray}\scriptsize\insertframenumber}}}}

\title{Difficulty and Complexity of Introductory Programming Problems}

\author{Tom\'a\v{s} Effenberger \and \textbf{Jaroslav \v{C}ech\'ak} \and Radek Pel\'anek\\[10mm]
    %Masaryk University Brno\\
    %Czech Republic
    \includegraphics[width=.3\linewidth]{al-logo}
}

%\author{Tom\'a\v{s} Effenberger\\[4mm]
%%Masaryk University Brno\\
%%Czech Republic
%\includegraphics[width=.3\linewidth]{al-logo}\\[6mm]
%}

\newcommand{\img}[2]{
  \begin{center}
    \includegraphics[width=#1\linewidth]{figures/#2}
  \end{center}
}


\date{CSEDM 2019}

\begin{document}

\frame{\titlepage}

\begin{frame}
  \frametitle{Difficulty vs. Complexity}
  \begin{center}
      \emph{Difficulty} relates to the observed performance.
      % difficulty is measure from student-problem interactions
      % examples: failure rate, median solving time, and number of attempts
      
      \bigskip
      \emph{Complexity} is an intrinsic characteristic of problems.
      % complexity is independet of the context such as students solving it
      % can be computed from problem statement and solution
      % examples: number of lines, number of flow-of-control structures, and number of programming concepts, 
  \end{center}

\end{frame}

\begin{frame}
  \frametitle{Why bother?}
  \begin{itemize}
      \item problem sequencing
      % often problems are presented from least complex to most complex or from least difficult to most difficult, so we need to know the complexity and difficulty
      \item cold start problem of difficulty measures
      % before any student come, there is no data for difficulty measure; can complexity help in giving estimate on the difficulty after the students will come?
      \item fine tuning problem difficulty
      % we can change specific dimension of complexity to obtain desired difficulty; for example require more lines of code but keep the same programming concepts
      \item detecting anomalies among problems
      % anomalies are problems that have highly different complexity and observed difficulty; maybe they are more "puzzles" than normal problem or they are misplaced in the system.
  \end{itemize}
\end{frame}

\begin{frame}
\frametitle{Our Goal}
Explore relationships between:
\begin{itemize}
    \item complexity measures
    % which measures measure the same thing; is it always like that
    \item difficulty measures
    % which measures measure the same thing; is it always like that
    \item complexity and difficulty measures
    % is the a relation between complexity and difficulty; can complexity predict difficulty?
\end{itemize}
% we used correlations analysis and ()PCA) projections
\end{frame}

\begin{frame}
\frametitle{Our Data}
\begin{table}[bt]
    \centering
    \begin{tabular}{l l r r r}
        \toprule
        Exercise & Interface & Problems & Students & Attempts  \\
        \midrule
        RoboMission & blocks & 85 & 3,800 & 62,500  \\
        Turtle Blockly & blocks & 77 & 11,000 & 63,600  \\
        Turtle Python & text & 51 & 2,400 & 11,900  \\
        Python & text & 73 & 2,000 & 10,700  \\
        \bottomrule
    \end{tabular}
    \label{tbl:data}
\end{table}
% block-based vs text-based; robot on a grid vs turtle graphics vs problem solving with python (sequences, lists, strings)
% 4 varying exercises (not just one) so we can test wheter or not our finding generalise
\end{frame}

\begin{frame}
\frametitle{Used Measures}
% we used alot of them for comparison, some make more sense some are more just to explore and make sure it is really the same
\begin{itemize}
    \item complexity
    \begin{itemize}
        \item instruction length
        % number of words / characters in the problem statement
        \item code length
        % number of lines / characters in the author's solution
        \item number of unique concepts
        % 3 levels of granularity; hand-made for each exercise
    \end{itemize}
    \item difficulty
    \begin{itemize}
        \item failure rate
        % unsucessful students / all students that have seen the problem
        \item median solving time
        % computed from sucessful students
        \item median number of attempts
        % attempts is and act of student submitting the solution for an automated assessment
    \end{itemize}
    % each difficulty measure also has additional variants obtained from filtering some students; students with less than k visited probelms; taking only attempts from kth visited problem on; k = 3, 5, and 10
\end{itemize}

\end{frame}

\begin{frame}
\frametitle{Methods}
\begin{itemize}
    \item Spearman's correlation
    \begin{itemize}
        \item comparing pairs of measures
        \item more indepth
    \end{itemize}
    \item Principal Component Analysis (PCA)
    \begin{itemize}
        \item broader overview
        \item multiple measures at the same time
    \end{itemize}
\end{itemize}

\end{frame}


\begin{frame}
\frametitle{Spearman's Correlation Example}
\img{.6}{correlation-robomission}
% number of concepts version small logical groups; failure rate without any filtering; each dot is a problem; color is level (darker less complex; britger more complex)
\end{frame}

\begin{frame}
\frametitle{Spearman's Correlation Examples}
\img{1}{correlations}
% illustrates than conclusions drawn from single exercise may not generalize to others; Turtle Python (0.82) vs Python (0.17)
\end{frame}

\begin{frame}
\frametitle{Spearman's Correlation Conclusion}
\begin{itemize}
    \item conclusions drawn from a single exercise may not generalize
    % as illustrated on the previous slide
    \item variants of the same measure correlate well ($r \ge 0.7$)
    % accross all exercises; exception concepts and failure rate in Python
\end{itemize}
% illustrates than conclusions drawn from single exercise may not generalize to others; Turtle Python (0.82) vs Python (0.17)
\end{frame}

\begin{frame}
\frametitle{PCA Example}
\img{.6}{pca-robomission}
% each point is a single variant of a measure; difficulty vs complexity
\end{frame}

\begin{frame}
\frametitle{PCA Results}
\img{1}{pca}
% variants of the same measure form close clusters (consistent with Spearman); exception is failure rate in Python
% median attempts in Python in due to different design choice
\end{frame}

\begin{frame}
\frametitle{PCA Conclusion}
\begin{itemize}
    \item variants of the same measure form tight clusters
    \item the same type of measures tend to be closer together
\end{itemize}
\end{frame}

\begin{frame}
\frametitle{Complexity for Difficulty Estimation}
% in our paper we all looked into estimating difficulty using only information bout complexity of the problem
\begin{itemize}
    \item using lines of code and number of concepts
    \item estimate solve time, number of attempts, and failure rate
\end{itemize}
\end{frame}

\begin{frame}
\frametitle{Conclusion}
\begin{itemize}
    \item conclusions drawn from a single exercise may not generalize
    \item filtering resulted in highly similar measures
    \item predicting difficulty using complexity is difficult
\end{itemize}
\end{frame}

\end{document}
