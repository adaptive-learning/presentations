\documentclass[bigger]{beamer}

\usepackage{booktabs}
\useinnertheme{rounded}
\usecolortheme{crane}
\setbeamerfont{block title}{size={}}

\title{Conceptual Issues in Mastery Criteria: Differentiating Uncertainty and
  Degrees of Knowledge}

\author{Radek Pel\'anek\\[10mm]
%Masaryk University Brno\\
%Czech Republic
\includegraphics[width=.3\linewidth]{al-logo}
}

\newcommand{\img}[2]{
  \begin{center}
    \includegraphics[width=#1\linewidth]{#2}
  \end{center}
}


\date{AIED 2018}

\begin{document}

\frame{\titlepage}

\begin{frame}
  \frametitle{Mastery Learning}

  TODO img
  \begin{itemize}
  \item student performance
  \item student modeling
  \item mastery criterion
  \end{itemize}
\end{frame}

\begin{frame}
  \frametitle{Threshold Criterion}

  \[ \theta > 0.95 \]

  What does this mean?

  \begin{itemize}
  \item portion of the topic that the learner mastered?
  \item uncertainty of the estimate?
  \end{itemize}
\end{frame}

\begin{frame}
  \frametitle{Bayesian Knowledge Tracing}

  \img{.5}{bkt}

  \begin{itemize}
  \item threshold on uncertainty
  \item binary knowledge assumption
  \end{itemize}
\end{frame}

\begin{frame}
  \frametitle{Logistic Models}

  \img{.5}{logistic}

  \begin{itemize}
  \item degrees of knowledge
  \item uncertainty of estimate not explicitly quantified
  \end{itemize}
\end{frame}

\begin{frame}
  \frametitle{LogisticHMM}

  \begin{itemize}
  \item generalization of BKT and logistic models
  \item goal of the model:
    \begin{itemize}
    \item clarification of conceptual issues: uncertainty vs degrees of
      knowledge
    \item not practical modeling
    \end{itemize}
  \end{itemize}
\end{frame}

\begin{frame}
  \frametitle{LogisticHMM}

  \img{.8}{logistichmm-intuition}
\end{frame}

\begin{frame}
  \frametitle{Emmision Probabilities}

  \img{.7}{hmm-model-examples1}
\end{frame}

\begin{frame}
  \frametitle{LogisticHMM and BKT}

  \img{.7}{hmm-model-examples2}
\end{frame}

\begin{frame}
  \frametitle{Using the Model}

  \img{}{hmm-model-demo1}
\end{frame}

\begin{frame}
  \frametitle{Experiments}

  Experiments with simulated data, generated by the LogisticHMM ...

  Comparision with other mastery criteria

  \begin{itemize}
  \item N consecutive correct
  \item Exponential moving average
  \item BKT
  \end{itemize}
\end{frame}

\begin{frame}
  \frametitle{Comparison with Simple Criteria}

  \begin{itemize}
  \item N consecutive correct
    \begin{itemize}
    \item difference when...
    \end{itemize}
  \item Exponential moving average
    \begin{itemize}
    \item similar performance possible...
    \item not clear how to set parameters... exp. weight, threshold...
    \end{itemize}
  \end{itemize}
\end{frame}

\begin{frame}
  \frametitle{Comparison with BKT}

  BKT fit...

  \img{.5}{hmm_bkt}
\end{frame}

\begin{frame}
  \frametitle{Consequences for Practice}

  the point is not that we should use LogisticHMM

  differentiate uncertainty and degrees of knowledge

  simple criteria may be sufficient ... number of attempts, average recent
  performance... with suitable thresholds...
\end{frame}

\begin{frame}
  \frametitle{Other Issues and Future Work}

  \begin{itemize}
  \item wheel-spinning students -- unable to master a topic
  \item relation to more complex student models
  \item multiple skills, forgetting, ...
  \end{itemize}

  \alert{mastery criteria are important and underexplored}
\end{frame}

\end{document}